\documentclass[a4paper,12pt]{article}
\usepackage{multicol}
\usepackage{calc}
\usepackage{ifthen}
\usepackage{hyperref}
\usepackage{multicol}
\usepackage{calc}
\usepackage{ifthen}
\usepackage[landscape]{geometry}
\usepackage{hyperref}
\usepackage{amsmath}
\usepackage{amssymb}
\usepackage{xcolor}
\usepackage{colortbl}
\usepackage{array}
\usepackage{enumitem}

% To make this come out properly in landscape mode, do one of the following
% 1.
%  pdflatex latexsheet.tex
%
% 2.
%  latex latexsheet.tex
%  dvips -P pdf  -t landscape latexsheet.dvi
%  ps2pdf latexsheet.ps


% If you're reading this, be prepared for confusion.  Making this was
% a learning experience for me, and it shows.  Much of the placement
% was hacked in; if you make it better, let me know...


% 2008-04
% Changed page margin code to use the geometry package. Also added code for
% conditional page margins, depending on paper size. Thanks to Uwe Ziegenhagen
% for the suggestions.

% 2006-08
% Made changes based on suggestions from Gene Cooperman. <gene at ccs.neu.edu>


% To Do:
% \listoffigures \listoftables
% \setcounter{secnumdepth}{0}


% This sets page margins to .5 inch if using letter paper, and to 1cm
% if using A4 paper. (This probably isn't strictly necessary.)
% If using another size paper, use default 1cm margins.
\ifthenelse{\lengthtest { \paperwidth = 11in}}
	{ \geometry{top=.5in,left=.5in,right=.5in,bottom=.5in} }
	{\ifthenelse{ \lengthtest{ \paperwidth = 297mm}}
		{\geometry{top=1cm,left=1cm,right=1cm,bottom=1cm} }
		{\geometry{top=1cm,left=1cm,right=1cm,bottom=1cm} }
	}

% Turn off header and footer
\pagestyle{empty}
 

% Redefine section commands to use less space
\makeatletter
\renewcommand{\section}{\@startsection{section}{1}{0mm}%
                                {-1ex plus -.5ex minus -.2ex}%
                                {0.5ex plus .2ex}%x
                                {\normalfont\large\bfseries}}
\renewcommand{\subsection}{\@startsection{subsection}{2}{0mm}%
                                {-1explus -.5ex minus -.2ex}%
                                {0.5ex plus .2ex}%
                                {\normalfont\normalsize\bfseries}}
\renewcommand{\subsubsection}{\@startsection{subsubsection}{3}{0mm}%
                                {-1ex plus -.5ex minus -.2ex}%
                                {1ex plus .2ex}%
                                {\normalfont\small\bfseries}}


% Don't print section numbers
\setcounter{secnumdepth}{0}


\setlength{\parindent}{3pt}
\setlength{\parskip}{0pt plus 0.2ex}

\begin{document}



\begin{multicols}{2}

% multicol parameters
% These lengths are set only within the two main columns
%\setlength{\columnseprule}{0.25pt}
\setlength{\premulticols}{1pt}
\setlength{\postmulticols}{1pt}
\setlength{\multicolsep}{1pt}
\setlength{\columnsep}{1pt}

\section{Improper Integrals (Convergence Tests)}

\subsection{P-Test}

\begin{itemize}
    \setlength\itemsep{0em}
    \item (3.2,q3.5) $\displaystyle\int _{0}^b\frac{dx}{x^p}$  and $\displaystyle\int _{a}^b\frac{dx}{(x-a)^p}$ and $\displaystyle\int _{a}^b\frac{dx}{(b-a)^p}$ converges iff $p<1$
    \item (3.12) if $a>0$ then $\displaystyle\int ^\infty_{a} \frac{dx}{x^p}$ converges iff $p>1$
    \item $\displaystyle\int ^\infty_{1}\frac{dx}{x^{p}} \, dx=\begin{cases} \frac{1}{p-1} &  p>1 \\ \text{diverges} & {p\leq 1}\end{cases}$
\end{itemize}


\subsection{Cauchy's Criterion}
(3.4) Let $f(x)$ be defined on $(a,b]$ and integrable on every interval $[t,b]$ (where $a<t<b$). Then:
\begin{itemize}
    \setlength\itemsep{0em}
    \item $\displaystyle\int ^b_{a}f(x) \, dx$ converges iff $\forall\varepsilon>0,\exists \delta>0:\forall r,s\in[a,a+\delta],\,\displaystyle\left|\int ^s_{r}f(x) \, dx\right|<\varepsilon$
\end{itemize}
(3.15) Let $f(x)$ be defined on $[a,\infty)$ and integrable on every interval $[a,t]$ (where $a<t$). Then:
\begin{itemize}
    \item $\displaystyle\int ^\infty_{a}f(x) \, dx$ conv. iff $\forall\varepsilon>0,\exists M>a:\forall r,s\in[M,\infty),\,\,\displaystyle\left|\int ^s_{r}f(x) \, dx\right|<\varepsilon$
\end{itemize}

\small(similar for $[a,b)$ and $(-\infty,b]$)
\normalsize

\subsection{Comparison Test $(a,b]$}
Let $f,g$ be non-negative functions defined on $(a,b]$ and integrable on every closed subinterval of $(a,b]$. 

\subsubsection{Direct Comparison Test}

(3.5) If $\exists \delta>0:\forall x\in(a,a+\delta),\,0\leq f(x)\leq g(x)$ then:

\begin{itemize}
    \item if $\int ^b_{a} g$ converges, then $\int ^b_{a} f$ converges
    \item if $\int ^b_{a} f$ diverges, then $\int ^b_{a} g$ diverges
\end{itemize}

\small(similar test for $[a,b)$)
\normalsize

\subsubsection{Limit Comparison Test}

\small
(3.5\*) Given $\displaystyle\lim_{ x \to a^+ }\frac{f(x)}{g(x)}=L$ exists, then:
\normalsize
\begin{itemize}
    \setlength\itemsep{0em}
	\item if $0<L<\infty$ then $\displaystyle\int ^b_{a}g$ converges iff $\displaystyle\int ^b_{a}f$ converges
	\item if $L=0$ and $\displaystyle\int ^b_{a}g$ converges, then $\displaystyle\int ^b_{a}f$ converges
	\item if $L=\infty$ and $\displaystyle\int ^b_{a}g$ diverge, then $\displaystyle\int ^b_{a}f$ diverge
	\item (similar test for $[a,b)$)
\end{itemize}


\subsection{Comparison Tests $[a,∞)$}

$f,g$ non-negative on $[a,\infty)$ and integrable on every closed subinterval $[a,t]\subset [a,\infty)$.

\subsubsection{Direct Comparison Test}
(3.16) If $\exists A\geq a:\forall x\in[A,\infty),\,0\leq f(x)\leq g(x)$ then:

\begin{itemize}
    \item if $\int ^\infty_{a} g$ converges, then $\int ^\infty_{a} f$ converges
	\item if $\int ^\infty_{a} f$ diverges, then $\int ^\infty_{a} g$ diverges
\end{itemize}

\subsubsection{Limit Comparison Test}

(3.16-\*) Given $\displaystyle\lim_{ x \to \infty }\frac{f(x)}{g(x)}=L$ exists (finite or infinite) then:

\begin{itemize}
    \setlength\itemsep{0em}
    \item if $0<L<\infty$ then $\int ^b_{a}g(x) \, dx$ converges if and only if $\int ^b_{a}g(x) \, dx$ converges
	\item if $L=0$ and $\int ^b_{a}g(x) \, dx$ converges then $\int ^b_{a}g(x) \, dx$ converges
	\item if $L=\infty$ and $\int ^b_{a}g(x) \, dx$ diverge then $\int ^b_{a}g(x) \, dx$ diverge
\end{itemize}

% TODO: 
% - Dirichlet's Test
% - Integral Test
% - Absolutely Integrable Function


\end{multicols}
\end{document}
