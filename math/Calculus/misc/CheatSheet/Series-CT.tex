\documentclass[a4paper,12pt]{article}
\usepackage{multicol}
\usepackage{calc}
\usepackage{ifthen}
\usepackage{hyperref}
\usepackage{multicol}
\usepackage{calc}
\usepackage{ifthen}
\usepackage[landscape]{geometry}
\usepackage{hyperref}
\usepackage{amsmath}
\usepackage{amssymb}
\usepackage{xcolor}
\usepackage{colortbl}
\usepackage{array}
\usepackage{enumitem}

% To make this come out properly in landscape mode, do one of the following
% 1.
%  pdflatex latexsheet.tex
%
% 2.
%  latex latexsheet.tex
%  dvips -P pdf  -t landscape latexsheet.dvi
%  ps2pdf latexsheet.ps


% If you're reading this, be prepared for confusion.  Making this was
% a learning experience for me, and it shows.  Much of the placement
% was hacked in; if you make it better, let me know...


% 2008-04
% Changed page margin code to use the geometry package. Also added code for
% conditional page margins, depending on paper size. Thanks to Uwe Ziegenhagen
% for the suggestions.

% 2006-08
% Made changes based on suggestions from Gene Cooperman. <gene at ccs.neu.edu>


% To Do:
% \listoffigures \listoftables
% \setcounter{secnumdepth}{0}


% This sets page margins to .5 inch if using letter paper, and to 1cm
% if using A4 paper. (This probably isn't strictly necessary.)
% If using another size paper, use default 1cm margins.
\ifthenelse{\lengthtest { \paperwidth = 11in}}
	{ \geometry{top=.5in,left=.5in,right=.5in,bottom=.5in} }
	{\ifthenelse{ \lengthtest{ \paperwidth = 297mm}}
		{\geometry{top=1cm,left=1cm,right=1cm,bottom=1cm} }
		{\geometry{top=1cm,left=1cm,right=1cm,bottom=1cm} }
	}

% Turn off header and footer
\pagestyle{empty}
 

% Redefine section commands to use less space
\makeatletter
\renewcommand{\section}{\@startsection{section}{1}{0mm}%
                                {-1ex plus -.5ex minus -.2ex}%
                                {0.5ex plus .2ex}%x
                                {\normalfont\large\bfseries}}
\renewcommand{\subsection}{\@startsection{subsection}{2}{0mm}%
                                {-1explus -.5ex minus -.2ex}%
                                {0.5ex plus .2ex}%
                                {\normalfont\normalsize\bfseries}}
\renewcommand{\subsubsection}{\@startsection{subsubsection}{3}{0mm}%
                                {-1ex plus -.5ex minus -.2ex}%
                                {1ex plus .2ex}%
                                {\normalfont\small\bfseries}}


% Don't print section numbers
\setcounter{secnumdepth}{0}


\setlength{\parindent}{3pt}
\setlength{\parskip}{0pt plus 0.2ex}

\begin{document}



\begin{multicols}{2}

% multicol parameters
% These lengths are set only within the two main columns
%\setlength{\columnseprule}{0.25pt}
\setlength{\premulticols}{1pt}
\setlength{\postmulticols}{1pt}
\setlength{\multicolsep}{1pt}
\setlength{\columnsep}{1pt}

\section{Series (Convergence Tests)}

\renewcommand{\arraystretch}{2} % Adjust the value to increase row spacing


\begin{tabular}{@{}ll@{}l@{}}
\textbf{Divergence Test} & $\displaystyle\lim_{n\to\infty} a_n \neq 0 \implies \sum a_n$ diverges \\
\textbf{Geometric Series} & $\displaystyle\sum ar^n = \frac{a}{1-r}$ (if $|r|<1$) \\
\textbf{P-Series} & $\displaystyle\sum_{n=1}^{\infty} \frac{1}{n^p}$ converges (if $p>1$) \\
\textbf{Integral Test} & $\displaystyle\int_{1}^{\infty} f(x) \, dx$ and $\displaystyle\sum_{n=1}^{\infty} f(n)$ converge together \\
\textbf{Cauchy C.T.} & $\displaystyle\sum a_n$ conv. iff \\
& $\forall \varepsilon > 0, \exists N: \forall n \geq N, \forall p \in \mathbb{N}, \sum_{k=n+1}^{n+p} a_k < \varepsilon$ \\
\textbf{Absolute C.T.} & $\sum |a_n|$ converges $\implies \sum a_n$ converges \\
\textbf{MCT} \\

\end{tabular}


\textbf{Direct Comparison} ($\forall n \geq N, 0 \leq a_n \leq b_n$) 
\begin{itemize}
\setlength\itemsep{0em}

\item $\sum b_n$ converges $\implies \sum a_n$ converges
\item $\sum a_n$ diverges $\implies \sum b_n$ diverges (both to infinity)
\end{itemize}

\textbf{Limit Comparison} 
($\lim_{n\to\infty} \frac{a_n}{b_n} = c > 0$)
\begin{itemize}
\setlength\itemsep{0em}

\item $\sum b_n$ converges $\iff \sum a_n$ converges
\item  $\sum b_n$ diverges $\iff \sum a_n$ diverges 
\end{itemize}

\subsubsection{Root Test (Cauchy)}

\begin{itemize}
\setlength\itemsep{0em}

\item (5.16a) If there exists $q<1$ such that $\sqrt[n]{|a_n|}\leq q$ for almost all $n$, then the series $\sum a_n$ converges absolutely. 
\item (5.16b) If $\sqrt[n]{|a_n|}\geq 1$ for infinitely many $n$, then $\sum a_n$ diverges.

\item (5.16*,**) Given $c=\displaystyle\limsup_{n\to\infty}\sqrt[n]{|a_n|}$. or ($c=\displaystyle\lim_{n\to\infty}\sqrt[n]{|a_n|}$ exists) \\
    $c<1\implies\sum a_n$ converges absolutely.\\
    $c>1\implies\sum a_n$ diverges.
\end{itemize}

\subsubsection{Ratio Test (d'Alembert)}
Given $a_n\neq0$ for all $n$

\begin{itemize}
\setlength\itemsep{0em}

\item (5.17a) If there exists $q<1$ such that $\left|\frac{a_{n+1}}{a_n}\right|\leq q$ for almost all $n$, then the series $\sum a_n$ converges absolutely.
\item (5.17b) If $\left|\frac{a_{n+1}}{a_n}\right|\geq 1$ for almost all $n$, then the series $\sum a_n$ diverges.    
\item (5.17**) Given $c=\lim_{n\to\infty}\left|\frac{a_{n+1}}{a_n}\right|$.
    \\ $c<1\implies\sum a_n$ converges absolutely.
    \\ $c>1\implies\sum a_n$ diverges.
\item (q5.26) If there exists $q<1$ such that $\left|\frac{a_{n+1}}{a_n}\right|\leq q<1$ for almost all $n$, then there exists $q\leq q'<1$ such that $\sqrt[n]{|a_n|}\leq q'$ for almost all $n$.
\end{itemize}



\subsubsection{Cauchy Condensation Test}

(5.18) Let $(a_n)$ be a decreasing sequence of non-negative terms. 

\[
\displaystyle\sum_{n=1}^{\infty}a_n \text{ converges} \displaystyle\iff\sum_{n=1}^{\infty}2^na_{2^n} \text{ converges}
\]

\subsubsection{Alternating Series Test (Leibniz)}

(5.20) Let $(a_n)$ be a decreasing, null, (thus nonnegative) sequence. Then:

\begin{enumerate}[label=\Alph*.]
\setlength\itemsep{0em}

    \item The series $\sum_{n=1}^{\infty}(-1)^{n+1}a_n$ converges. (This is a special case of Dirichlet's test. see q5.32) 
\end{enumerate}

A.S. Est. - If $S=\sum (-1)^{n+1}a_n$, and $S_n$ is the $n$-th partial sum, then for all $n$ we have:

\begin{enumerate}[label=\Alph*. , start=2]
\setlength\itemsep{0em}

    \item $S$ is between $S_n$ and $S_{n+1}$.
    \item The error $|S-S_n|$ is less than $a_{n+1}$.
\end{enumerate}



% TODO: 
% - Dirichlet's Test
% - Integral Test
% - Absolutely Integrable Function


\end{multicols}
\end{document}
