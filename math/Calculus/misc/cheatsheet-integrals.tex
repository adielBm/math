\documentclass[a4paper,12pt]{article}
\usepackage{multicol}
\usepackage{calc}
\usepackage{ifthen}
\usepackage{hyperref}
\usepackage{multicol}
\usepackage{calc}
\usepackage{ifthen}
\usepackage[landscape]{geometry}
\usepackage{hyperref}
\usepackage{amsmath}
\usepackage{amssymb}
\usepackage{xcolor}
\usepackage{colortbl}
\usepackage{array}
\usepackage{enumitem}

% To make this come out properly in landscape mode, do one of the following
% 1.
%  pdflatex latexsheet.tex
%
% 2.
%  latex latexsheet.tex
%  dvips -P pdf  -t landscape latexsheet.dvi
%  ps2pdf latexsheet.ps


% If you're reading this, be prepared for confusion.  Making this was
% a learning experience for me, and it shows.  Much of the placement
% was hacked in; if you make it better, let me know...


% 2008-04
% Changed page margin code to use the geometry package. Also added code for
% conditional page margins, depending on paper size. Thanks to Uwe Ziegenhagen
% for the suggestions.

% 2006-08
% Made changes based on suggestions from Gene Cooperman. <gene at ccs.neu.edu>


% To Do:
% \listoffigures \listoftables
% \setcounter{secnumdepth}{0}


% This sets page margins to .5 inch if using letter paper, and to 1cm
% if using A4 paper. (This probably isn't strictly necessary.)
% If using another size paper, use default 1cm margins.
\ifthenelse{\lengthtest { \paperwidth = 11in}}
	{ \geometry{top=.5in,left=.5in,right=.5in,bottom=.5in} }
	{\ifthenelse{ \lengthtest{ \paperwidth = 297mm}}
		{\geometry{top=1cm,left=1cm,right=1cm,bottom=1cm} }
		{\geometry{top=1cm,left=1cm,right=1cm,bottom=1cm} }
	}

% Turn off header and footer
\pagestyle{empty}
 

% Redefine section commands to use less space
\makeatletter
\renewcommand{\section}{\@startsection{section}{1}{0mm}%
                                {-1ex plus -.5ex minus -.2ex}%
                                {0.5ex plus .2ex}%x
                                {\normalfont\large\bfseries}}
\renewcommand{\subsection}{\@startsection{subsection}{2}{0mm}%
                                {-1explus -.5ex minus -.2ex}%
                                {0.5ex plus .2ex}%
                                {\normalfont\normalsize\bfseries}}
\renewcommand{\subsubsection}{\@startsection{subsubsection}{3}{0mm}%
                                {-1ex plus -.5ex minus -.2ex}%
                                {1ex plus .2ex}%
                                {\normalfont\small\bfseries}}


% Don't print section numbers
\setcounter{secnumdepth}{0}


\setlength{\parindent}{3pt}
\setlength{\parskip}{0pt plus 0.2ex}

\begin{document}



\begin{multicols}{2}

% multicol parameters
% These lengths are set only within the two main columns
%\setlength{\columnseprule}{0.25pt}
\setlength{\premulticols}{1pt}
\setlength{\postmulticols}{1pt}
\setlength{\multicolsep}{1pt}
\setlength{\columnsep}{1pt}

\section{Integrals}
\begin{tabular}{@{}ll@{}l@{}}
Linearity & $\int (\alpha f+\beta g)=\alpha\int f +\beta \int g$ \\
Power Rule & $\int x^r \, dx=\frac{x^{r+1}}{r+1}+C$ & $r\neq -1$ \\
& $\int \frac{1}{x} \, dx=\ln|x|+C$ & \\
Exponential & $\int a^x \, dx=\frac{a^x}{\ln a}+C$ & $a>0, a\neq 1$ \\
& $\int e^x \, dx=e^x+C$ & \\

\end{tabular}



\subsection{Integration by Parts}

\large
\[
\int f(x)g'(x) \, dx=f(x)g(x) -\int f'(x)g(x) \, dx
\] 
\[
\int u \, dv = uv - \int v \, du
\]
\subsection{Substitution}

\renewcommand{\arraystretch}{1} % Adjust the value to increase row spacing

\[
\int f(g(x))g'(x)\, dx = {\color{gray}\left[\begin{array}{rl}
u&= g(x) \\ du&= g'(x)dx \end{array}\right]} = \int f(u) \, du 
\]
\normalsize

Return to $x$ by substitute $u=g(x)$

\small
(Assumptions: $f$ is continuous on $I$, $g$ is continuously differentiable on $J$, $g(J) \subseteq I$ (i.e. image of $g$ is subset of $I$, so that $f \circ g$ is defined))


\normalsize

\subsubsection{Logarithmic Integration}
\[
\int \frac{f'(x)}{f(x)} \, dx = \ln|f(x)| + C
\]

% \small (special case of substitution with $u = f(x)$ and $du = f'(x)dx$)

\subsection{Substitution V2}
\[
\int f(x) \, dx = {\color{gray}\left[\begin{array}{rl} x &= \varphi(t) \\ dx &= \varphi'(t)dt \end{array}\right]} = \int f(\varphi(t))\varphi'(t) \, dt 
\]

Return to $x$ by substitute $t=\varphi^{-1}(x)$


\subsection{2.5}

$\displaystyle\int f(x) \, dx=F(x)+C\implies\displaystyle\int f(\alpha x+\beta) \, dx=\frac{1}{\alpha}F(\alpha x+\beta)+C$ 



\subsection{Trigonometric Substitution} 

\subsection{Formulas}

\begin{tabular}{@{}ll@{}l@{}}



ln & $\int \ln x \, dx=x\ln x-x+C$ \\


Trigonometric & $\int \sin x \, dx=-\cos x+C$ \\
& $\int \cos x \, dx=\sin x+C$ \\
& $\int \tan x \, dx=-\ln|\cos x|+C$ \\
& $\int \cot x \, dx=\ln|\sin x|+C$ \\
Reciprocal Trig. & $\displaystyle\int \frac{1}{\sin x} \, dx=\ln\left|\tan \frac{x}{2}\right|+C$ \\
& $\displaystyle\int \frac{1}{\cos x} \, dx=-\ln\left|\tan \left( \frac{x}{2}-\frac{\pi}{4} \right)\right|+C$ \\
& $\displaystyle\int \frac{1}{\sin^2x} \, dx=-\cot x$ \\
& $\displaystyle\int \frac{1}{\cos^2x} \, dx=\tan x$ \\

Inverse Trig. &  $\displaystyle\int \arcsin x \, dx=x\arcsin x+\sqrt{ 1-x^2 }+C$ \\	

& $\displaystyle\int \arctan x \, dx=x\arctan x-\frac{1}{2}\ln(x^2+1)+C$ \\
 
Der. of Inverse Trig. & $\displaystyle\int \frac{1}{1 + x^2} \, dx = \arctan x + C$ \\
& $\displaystyle\int \frac{1}{\sqrt{1 - x^2}} \, dx = \arcsin x + C$ \\
& $\displaystyle\int \frac{1}{\sqrt{ a^2+x^2 }} \, dx=\arcsin \frac{x}{a}+C$ \\



(2.8a) & $\displaystyle\int x\ln x \, dx=\frac{x^2\ln x}{2}-\frac{x^2}{4}+C$ \\
(2.9) & $\displaystyle\int \ln x \, dx=x\ln x-x+C$ \\
(2.10) & $\displaystyle\int \ln^2x \, dx=x\ln^2x-2x\ln x+2x+C$ \\
% (e2.12) & $\displaystyle I_{m}=\int \frac{1}{(x^2+a^2)^{m}} \, dx={\begin{cases}I_{1}=\frac{1}{a}\arctan \frac{x}{a}+C \\ I_{m+1}=\frac{1}{2ma^2}\cdot \frac{x}{(x^2+a^2)^m}+\frac{2m-1}{2ma^2}I_{m} \end{cases}}$ \\
(by power rule) & $\int \frac{1}{\sqrt{ x }} \, dx=2\sqrt{ x }+C$ \\
& $\int 0 \, dx=C$ \\
& $\int 1 \, dx=\int  \, dx=x+C$ \\

\end{tabular}


\newpage

\section{Definite Integrals}

\begin{tabular}{@{}ll@{}l@{}}   
Additivity & $\int_{a}^{b} f(x) \, dx = \int_{a}^{c} f(x) \, dx + \int_{c}^{b} f(x) \, dx$ &  $a<c<b$ and $f$ is int. on $[a,b]$ \\
Shift Property & $\int_{a}^{b} f(x) \, dx = \int_{a-c}^{b-c} f(x+c) \, dx$ \\
& $\int_{0}^{a} f(x) \, dx = \int_{0}^{a} f(a-x) \, dx$ \\
& $\int_{a}^{b} c \, dx = c(b-a)$ \\
& $\int_{a}^{a} f(x) \, dx = 0$ \\


\end{tabular}


\end{multicols}
\end{document}
